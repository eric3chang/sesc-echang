%% LyX 1.6.9 created this file.  For more info, see http://www.lyx.org/.
%% Do not edit unless you really know what you are doing.
\documentclass[letterpaper,english]{article}
\usepackage[T1]{fontenc}
\usepackage[latin1]{inputenc}
\usepackage{array}
\usepackage{amsmath}
\usepackage{graphicx}
\usepackage{amssymb}

\makeatletter

%%%%%%%%%%%%%%%%%%%%%%%%%%%%%% LyX specific LaTeX commands.
\special{papersize=\the\paperwidth,\the\paperheight}

%% Because html converters don't know tabularnewline
\providecommand{\tabularnewline}{\\}

%%%%%%%%%%%%%%%%%%%%%%%%%%%%%% User specified LaTeX commands.
% This file was converted to LaTeX by Writer2LaTeX ver. 1.0.2
% see http://writer2latex.sourceforge.net for more info

\usepackage{amsfonts}\@ifundefined{definecolor}
 {\usepackage{color}}{}
\usepackage{array}\usepackage{supertabular}\usepackage{hhline}\usepackage{hyperref}\hypersetup{colorlinks=true, linkcolor=blue, citecolor=blue, filecolor=blue, urlcolor=blue}
% Outline numbering

\renewcommand{\thesection}{\arabic{section}}
\renewcommand{\thesubsection}{\arabic{section}.\arabic{subsection}}
\renewcommand{\thesubsubsection}{\arabic{section}.\arabic{subsection}.\arabic{subsubsection}}
\makeatletter
\newcommand{\arraybslash}{\let\\\@arraycr}
\makeatother
% Page layout (geometry)
\setlength{\voffset}{-1in}
\setlength{\hoffset}{-1in}
\setlength{\topmargin}{1in}
\setlength{\oddsidemargin}{1in}
\setlength{\textheight}{9.036833in}
\setlength{\textwidth}{6.5in}
\setlength{\footskip}{26.148pt}
\setlength{\headheight}{0cm}
\setlength{\headsep}{0cm}
% Footnote rule
\setlength{\skip\footins}{0.0469in}
\renewcommand{\footnoterule}{\vspace*{-0.0071in}\setlength\leftskip{0pt}\setlength\rightskip{0pt plus 1fil}\noindent\textcolor{black}{\rule{0.25\columnwidth}{0.0071in}}\vspace*{0.0398in}}
% Pages styles
\makeatletter
\newcommand{\ps@Standard}{
  \renewcommand\@oddhead{}
  \renewcommand\@evenhead{}
  \renewcommand\@oddfoot{\thepage{}}
  \renewcommand\@evenfoot{\@oddfoot}
  \renewcommand\thepage{\arabic{page}}
}
\makeatother

\setlength{\tabcolsep}{1mm}
\renewcommand{\arraystretch}{1.3}
\newcounter{Figure}
\renewcommand{\theFigure}{\arabic{Figure}}
\title{}

\makeatother

\usepackage{babel}

\begin{document}

\title{\textbf{An Analysis of Directory-Based Cache-Coherence Protocols
on Multiprocessors Using the SESC: SuperESCalar Simulator}}


\author{Eric Chang}

\maketitle
\noindent \textbf{Abstract}

\noindent \textit{Directory-based cache-coherence protocols are based
on the central concept of having the states of any particular cache
block located in a known, fixed, location. This report will discuss
the advantages and disadvantages of two different directory-based
cache-coherence protocols. I will introduce various cache-coherence
protocols in general. Then, I will introduce the specific protocols
that will be compared, which are the regular directory-based cache-coherence
protocol where all requests and responses must pass through the home
node, and a different protocol based on the protocol used in the SGI
Origin 2000-based systems. I will then go into detail about simulating
and implementing the differences between the protocols using the SESC:
SuperESCalar Simulator. In addition, results from simulations using
SPLASH2 benchmarks are presented. We show that the Origin directory-based
cache-coherence protocol has several advantages over the regular directory-based
cache-coherence protocol.}


\section[Introduction]{Introduction}

Computers today are moving increasingly towards multiprocessing architectures
because we have reached a thermal barrier in increasing transistor
switching speeds. As such, it is important to figure out ways to implement
effective multiprocessing architectures and how schemes offer different
tradeoffs between speed and memory consistency. In particular, the
cache-coherence schemes used for communication by processors have
a huge impact on the performance of a multiprocessing architecture.

Cache-coherence protocols require mechanisms like snoopy buses and
directories to function. It is important to understand how underlying
mechanisms such as buses and directories allow multiple caches to
communicate with each other. {}``In a coherent multiprocessor, the
caches provide both migration and replication of shared data items''
\cite{HEN00}. It is important for the architect of the processor
to design these features into the processor as to take advantage of
the speedup available in having multiple blocks of the same data across
different nodes, while still giving the correct results.

The Origin 2000 system is not directly applicable to CMPs (chip multiprocessors),
but it was designed for older systems. In particular, this system
has a part of the main memory with every node, which is not the case
on a CMP. For the report, I am considering a CMP where the last-level
cache is distributed across the nodes (cores) and I am ignoring misses
from the last-level cache.

In the following section of the paper, the outlines for different
cache-coherence protocols are presented. Section 3 details the designs
for the two cache-coherence protocols that I will be comparing. The
implementation of the two protocols is presented in Section 4. Performance
of the two types of protocols is presented in Section 5. Section 6
describes some of the problems faced in simulating the protocols.
Finally, section 7 concludes the report.


\section[Background]{Background}

There are several different types of cache-coherence protocols, and
each of them provide their own advantages and disadvantages. I will
first introduce the states that the caches need to store to permit
cache-coherence protocols to function, then I will go into the differences
between the two major cache-coherence protocols that are in use today,
which are the snoopy-based cache-coherence protocol and the directory-based
cache-coherence protocol, as well as modifications to the directory-based
cache-coherence protocol based on the Origin protocol.


\subsection[Snoopy{}-Based Versus Directory{}-Based Cache{}-Coherence Protocols]{Snoopy-Based Versus Directory-Based Cache-Coherence Protocols}

The most important job of any cache-coherence protocols is to maintain
data coherency. In order to maintain data coherency, the protocols
all need to make sure that whenever a cache writes to its block, it
has exclusive access to it. Given that constraint, one can perform
myriad varieties of modifications to the cache-coherence protocols
to achieve the best performance. This section will go over the snoopy-based
cache-coherence protocol, the regular directory-based cache-coherence
protocol, and the Origin-based cache-coherence protocol that are presented
in \figurename\ \ref{fig:cache-coherence-protocols}. All of these
cache-coherence protocols can use the MESI protocol for their cache
portion. This means that they all can support the Clean Exclusive
state in their caches. Although the two directory-based cache-coherence
protocols that will be analyzed in detail in this report will both
be using the MESI protocol for their caches, the interactions they
have is still vastly different. 

%
\begin{figure}
\includegraphics[width=1\textwidth]{diagrams/cache-coherence-protocol-combined}

\caption{\textit{\emph{Top: snoopy-based cache-coherence protocol. Middle:
regular directory-based cache-coherence protocol. Bottom: SGI Origin-based
cache-coherence protocol.}}\label{fig:cache-coherence-protocols}}
%
\end{figure}


The snoopy-based cache-coherence protocols only logically have two
different nodes, the requesting node and the data node. Every message
is broadcast to all the nodes. In any directory-based cache-coherence
protocol, there are three logical nodes in any request: the requesting
node, the directory node, and the data node. These logical nodes can
all be the same physical nodes or they can all be different, but it
is easier to think of them logically as three separate nodes. Requesting
node is the node that sends out the original read or write request.
In a directory-based protocol, it is necessary to send the request
to the directory node in order to find out where the data actually
is, what state the directory block is in, and whether or not the request
can be satisfied immediately. In the Origin cache-coherence protocol,
there are also three logical nodes, but occasionally, the data node
needs to send two messages at once, one to the directory node, and
one to the requesting node. This should decrease the overall latency
in the system because it eliminates the time going from the data node
to the directory node to complete requests.


\subsubsection[Snoopy{}-Based Cache{}-Coherence Protocol]{Snoopy-Based Cache-Coherence Protocol}

There are two major categories of cache-coherence protocols available
today, snoopy-based cache-coherence protocol and directory-based cache-coherence
protocol. Although the simplicity of snoopy-based cache-coherence
protocols allow manufacturers to easily convert single-core processors
to be used as multiprocessors, snoopy-based cache-coherence protocols
have problems with scaling up to meet higher core counts than directory-based
cache-coherence protocols. For that reason, this report will be focusing
mainly on directory-based cache-coherence protocols.

The snoopy-based cache-coherence protocol relies on the bus to transfer
necessary information. \ All messages are broadcast on a common bus,
which is monitored by each processor and its cache; therefore, each
processor needs a controller to snoop the bus. Anytime a node encounters
an important message that pertains to itself, the controller makes
sure to forward the request to the cache so that it can further process
it. When the processor requires something, it sends its request on
the bus, and this broadcast message is heard by every processor connected
to the bus.

Snoopy-based protocols have an advantage when it comes to manufacturing
because they can use the existing bus to memory as the broadcast medium
for communicating information about cache coherence \cite{HEN00}.
However, a snoopy-based protocol is not as scalable as a directory-based
protocol. The main problem with scaling up snoopy-based protocols
is that since all the processors are sharing the same medium, the
bus, unless the bandwidth and speed of the bus can be scaled infinitely
high, the number of processors cannot be scaled arbitrarily high.
Although there are many cache-coherence protocols fundamentally based
on the snoopy-based cache-coherence protocol, this cache-coherence
protocol will not be presented in the final results because this report
is focused on comparing differences in directory-based cache-coherence
protocols.


\subsubsection{Directory-Based Cache-Coherence Protocol}

Directory-based cache-coherence protocols are a class of widely used
cache-coherence protocols that has been proven in the past to be able
to scale up compared to cache-coherence protocols based on the snoopy
method. {}``As processor speeds and the number of cores per processor
increase, more designers are likely to opt for {[}directory-based
cache coherence{]} protocols to avoid the broadcast limit of a snoopy
protocol'' \cite{HEN00}. Directory-based protocols are interesting
because it can achieve higher speeds performing the same amount of
work than snoopy-based protocols. Nevertheless, there are still some
characteristics of directory-based cache-coherence protocols that
can cause problems when designing a system. A snoopy-based protocol
has very few problems with consistency because all messages are broadcast
on a bus, meaning all processors know whenever changes occur in the
system. However, in a directory-based protocol, it is possible for
the directory to be in an inconsistent state for a longer time. For
example, when we are evicting a block from a processor, this is broadcast
across the bus in a snoopy-protocol, and each node knows right away
if it needs to change its cache state. In a directory-based cache-coherence
protocol, there is some time between when the eviction message is
sent from the owner to when the message arrives at the directory.
In this time, it is possible for some other request to come in to
the directory, allowing the directory to possibly fetch the invalidated
data from the owner that just evicted its data if care is not taken
when designing the protocol.

A directory-based cache-coherence protocol can be designed several
ways. The directory needs to keep track of which CPU has which cache
line. This is necessary since the directory acts as the communication
channel between each CPU. In a snoopy system, we do not keep track
of which CPU has which cache block in a centralized location. This
is the advantage in the directory-based cache-coherence protocol that
allows fewer messages to be sent. One of the easiest ways to keep
track of the directories is to keep a bit vector in each distributed
directory about which CPU has which cache block. Using bit vector
to keep track of the CPU that has a specific cache block is an efficient
way to implement directory-based cache-coherence protocol, but it
has a limit on how many CPUs can be used together with the system
since the bit vector would grow too large in a system with too many
CPUs. Another way would be to keep track of the node ID in the directory.
The advantage of using a node ID is that it can potentially take up
less space in the directory when we have a large number of processors,
since we do not have to keep track of all processors. However, there
is a disadvantage to keeping track of the node ID. Usually, in a node
ID based directory-based cache-coherence system, we do not keep track
of all the node IDs, because that would defeat the primary purpose
of using the node IDs, which is to save space. Therefore, when we
have a system that keeps track of the cache block using node ID, we
are unable to keep track of all the CPUs that might potentially request
for a cache block. In that case, we would have to invalidate a CPU
when we want more space in the directory.

The directory is required to store more than just which CPU currently
contains a cache line from the current directory. It is also necessary
for the directory to store the owners and sharers in order to find
out which processors own or share a block. If we did not store these
in the state of the directory, we would have to query all the processors
to find out what state they are all storing, which would be incredibly
inefficient. \tablename~\ref{tab:Correspondence-between-cache}
lists the possible data that the directory can hold compared to the
cache state in any cache that has the block.

%
\begin{table}
\begin{centering}
\begin{tabular}{|c|c|}
\hline 
Cache State  & Directory Data Structure \tabularnewline
\hline
\hline 
Modified (Dirty Exclusive) & Owner = \{P\}, Sharers = \{\}\tabularnewline
\hline 
Exclusive (Clean Exclusive) & Owner = \{P\}, Sharers = \{\}\tabularnewline
\hline 
Shared & Owner = \{P\}, Sharers = \{Any number of P\}\tabularnewline
\hline 
Invalid & Owner = \{\}, Sharers = \{\}\tabularnewline
\hline
\end{tabular}
\par\end{centering}

\caption{Correspondence between cache state and directory data structure\label{tab:Correspondence-between-cache}}
%
\end{table}


In both the {}``Modified'' state and the {}``Exclusive'' state,
the directory would store only the node ID of the processor that is
the owner. Nevertheless, the directory does not know for certain whether
or not the owner is actually clean or dirty until it has sent a request
to the cache. It is the job of the protocol to take advantage of having
the {}``Exclusive'' state by not writing the block to memory if
the block is clean. If the directory is in a {}``Shared'' state,
then it would have a node ID in the owner slot and any number of node
IDs in the sharers list. The reason that there would be a node ID
in the owner slot is because any transition into the {}``Shared''
state must have transitioned from a {}``Modified'' or {}``Exclusive''
state, which must already have stored a node ID in the owner slot.

Keeping track of the owner could be done by either adding a bit to
each directory entry, and turning on the extra bit to indicate ownership.
There could also be an extra entry dedicated to holding the owner.
The advantage of adding a bit to each directory entry that holds the
node ID is that it could potentially save bits if there are many nodes
in the system and each node ID list is short. However, using an extra
entry strictly for holding the owner is more intuitive, and it can
also be faster to access since the processor always know which entry
is the owner. We can also have a combined owner-sharers list, where
one entry implies {}``Exclusive,'' and multiple entries imply that
the directory is in the {}``Shared'' state. In this last method,
the system would have to be able to quickly scan through all the entries
it has to determine whether it has no valid entries, one valid entry,
or multiple valid entries. Since in either of the protocols that will
be tested in this report, the owner entry becomes essentially an additional
entry for the sharers list, there is not truly a need for the directory
to have different slots for holding the owner versus a list of sharers.
Therefore, sometimes in the discussion of protocols in this report,
sharers will imply both the sharers list and also the node ID in the
owner slot.


\subsubsection{Modifications to the Directory-Based Cache-Coherence Protocol}

There is an optimized version of the basic directory protocol mentioned
in the previous section. The SGI Origin 2000 implemented this protocol,
which is based on an altered version of the protocol used in the Stanford
DASH multiprocessor \cite{LAU00}. This protocol tries to be more
optimized by reducing latency between the data node and the directory
node. It achieves this by assuming that all read requests to the directory
can be satisfied, and if it cannot, it is up to the owner of the data
instead of the directory to send an additional invalidate back to
the directory as well as the data response to the original requesting
node. We save time by allowing the data packet retrieved to go directly
to the requester without the additional latency of going to the directory
first.

This type of protocol introduce many opportunities for the messages
to arrive out-of-order. A requesting message goes to the directory,
which can be different than the node that ultimately supplies the
response back to the requester. There are situations where the directory
completes its transition into a non-busy state before knowing for
certain that the new owner has received its request, making it possible
for the directory to send another request to the new owner when the
new owner has not even received its response from the previous transaction.
In addition, the system sometimes need to send out two messages simultaneously,
as opposed to the regular directory, where all operations can only
cause one outgoing message to be sent on every incoming message. The
protocol deals with this situation by having the directory change
to a busy state whenever the directory state and the request cannot
be satisfied immediately. This situation can happen, for example,
when the directory state is Exclusive with another owner, and a read
request comes in. Any further requests that arrive, regardless of
their origin or type, will be denied via a nak by the directory. This
ensures that the directory stays in a consistent state and is not
modified based on invalid directory state.


\subsection[Cache States]{Cache States}

There are several states that has to be stored by the cache in order
to enable the rest of the system to function properly. I will go over
the states stored in the MSI cache protocol first. This protocol allows
the least amount of information to be stored, yet it still provides
enough information for a cache to acquire exclusive access. An improvement
upon this protocol can be found in the MESI protocol, which uses one
more state than the MSI protocol.


\subsubsection{MSI Cache}

The MSI protocol is the most basic of the cache-coherence protocols,
using only three states to ensure cache coherency. The three states
stand for Modified(M), Shared(S), and Invalid(I). The Modified state
signifies that the cache block is in a Dirty Exclusive mode and that
no other cache contains the entry. This is necessary whenever the
processor needs to perform a write operation. The Shared state means
that the cache block can exist in caches other than the current one,
and Invalid means that there are no usable data in this particular
cache line.

Because this protocol contains only three states, it can save space
on storage as compared to some more advanced schemes. In addition,
the protocol for a cache-coherence protocol designed using less cache
states can be simpler. The disadvantage is that this protocol requires
more messages and higher latency on average as compared to protocols
where the cache utilizes more states.


\subsubsection{MESI Cache}

The MESI protocol is an improvement upon the MSI protocol in that
it adds an Exclusive(E) state to the protocol. Whenever a cache holds
a block in the Exclusive state, it means that the cache has Exclusive
access to the block, but had no intention of altering the block at
the time of request. In other words, the Exclusive state indicates
Clean Exclusive. Adding the Exclusive state is advantageous in that
it reduces the traffic caused by writes of blocks that only exist
in one cache. This state allows the protocol to send fewer messages
and achieve lower latency in certain operations. A disadvantage of
this protocol is that it requires more space to store this additional
information as compared to the MSI protocol.

Once we add this state into the protocol, certain cases that required
an additional request into the network in the MSI protocol becomes
an operation where no additional messages are emitted. For example,
the MSI protocol fills a Clean Exclusive reply as Shared, so it becomes
necessary to send an additional network message to request for exclusive
ownership when transitioning from Shared to Dirty Exclusive. In a
MESI cache, since it has the Clean Exclusive state, the same transition
would go from Clean Exclusive to Dirty Exclusive without emitting
any messages on the network. In addition, it is easy to see that adding
the Exclusive state means the cache does not have to write back its
block to memory when downgrading from Clean Exclusive to Shared. In
the MSI protocol, there is no way to distinguish between a Dirty Exclusive
and a Clean Exclusive, meaning there are situations where we will
write a clean block to memory.


\section{Cache-Coherence Protocols}

This section will present the specific operations of the two different
types of directory-based cache-coherence protocols. First, I will
be showing the regular directory-based cache-coherence protocol where
all requests pass through the home node and each node can only emit
one response in reply to a request. Then, I will be showing a more
optimized version of the cache-coherence protocol, based on the SGI
Origin protocol, where it is possible for each node to emit two messages
in response to a request, and not all requests have to pass through
the directory before being satisfied.


\subsection{Regular Protocol}

The regular directory-based cache-coherence protocol is a traditional
directory-based cache-coherence protocol where the requester first
sends a message to the directory, followed by the directory sending
another request to the data node, with the data node returning a response
to the directory. Finally, the directory is responsible for forwarding
the response back to the requester. In general, two bilateral interactions
occur in this protocol to form a complete transaction.

The basic directory protocol is easier to implement than some of the
more advanced and newer version of the protocol, but it is also unoptimized.
This protocol makes sure that each request has a corresponding reply
and that no operations can proceed until the responding message has
been received. No assumptions are made when deciding the directory
state because we wait for the owner or shared block to return to the
directory before we forward it back to the requester.


\subsubsection{Regular Protocol Directory FSM}

\figurename~\ref{fig:regular-protocol-directory-fsm} illustrates
the directory FSM for the regular directory-based cache-coherence
protocol. The directory for any address starts in the {}``Unowned''
state initially. When the directory receives a read request, it would
set requester as the owner, forward the request to the memory node,
and transition to {}``Exclusive Waiting for Reply.'' In that state,
the directory has to wait for a response from the memory node before
it can do anything else, so any read requests that arrive gets nakked.
No writeback requests can arrive in {}``Unowned'' or {}``Exclusive
Waiting for Reply,'' because no node has the block, yet. When the
directory receives a reply from the memory node in {}``Exclusive
Waiting for Reply,'' an exclusive reply is sent to the requester,
and the directory transitions into the {}``Exclusive'' state.

%
\begin{figure}
\includegraphics[width=1\textwidth]{diagrams/regular-directory-fsm-1}

\caption{\textit{\emph{Regular protocol directory FSM}}\label{fig:regular-protocol-directory-fsm}}
%
\end{figure}


From the {}``Exclusive'' state, there are a myriad of situations
that can occur. The first can happen when an exclusive read request
arrives at the directory and the requester is the owner. This situation
occurs when the cache has the block in its {}``Shared'' state, but
it wants exclusive access to it for a write. The directory has to
send a read ack (no data) to the requester when this happens. The
second can happen when a writeback request is emitted from the owner.
In this case, the directory will clear the owner list, send the write
to memory, send a writeback ack to the requester, and transition to
{}``Unowned.'' The third is when the directory receives an eviction
request in this state. The directory is required to send an eviction
ack to the requester, clear the owner list, and transition to {}``Unowned.''
It is not necessary to write to memory in this case, because an eviction
request indicates that the message came from a cache whose state was
{}``Clean Exclusive.''

Another type of transition that occurs from the {}``Exclusive''
state is when the directory receives an exclusive read and the requester
is not the owner. In this situation, the directory would set the requester
as the owner and send an invalidate to the previous owner, while transitioning
to {}``Waiting for Invalidate Response,'' where it waits for a writeback
request, an eviction request, or an invalidate ack to cause it to
transition back to {}``Exclusive.'' If a read request arrives while
the directory is in {}``Waiting for Invalidate Response,'' the directory
would send a nak to the requester. If the directory receives a writeback
request, it would send an exclusive reply to the new owner, a writeback
ack to the requester, send a write to memory, and transition to {}``Exclusive.''
On the other hand, if only an eviction request was received, the directory
would send an exclusive reply to the new owner, send an eviction ack
to the requester, and transition to {}``Exclusive'' without sending
a write to memory. In {}``Waiting for Invalidate Response,'' the
directory could also receive an invalidate response or an invalidate
ack. In both cases, the directory would send an exclusive reply to
the new owner and transition to {}``Exclusive,'' but it would only
send a write to memory if it received an invalidate response, indicating
that the cache was in the {}``Dirty Exclusive'' state.

It is also possible to transition from the {}``Exclusive'' state
to {}``Exclusive Shared Waiting for Reply'' when the directory receives
a shared read. This shared read has to come from a requester that
is not the owner. The directory has to forward the request to the
previous owner and add the requester as a sharer. In {}``Exclusive
Shared Waiting for Reply,'' several things can happen. If the directory
receives a read, it would send a nak to the requester. If a writeback
request arrives from the previous owner, then the directory would
set the owner to be the read requester, send a writeback ack to the
writeback requester, send an exclusive read reply to the read requester,
send a write to memory, and transition back to {}``Exclusive.''
Receiving an eviction request is similar to receiving a writeback
request, except the directory does not have to send a write to memory.
Otherwise, if it receives a read reply from the previous owner, then
the directory would forward a shared reply to the requester and transition
to {}``Shared.''

In the {}``Shared'' state, if the directory receives an eviction
request from a sharer and there are more than two sharers, then the
directory would simply remove the requester from the sharers list
and send an eviction ack to the requester. If the directory receives
a writeback request and the size of the sharers list is 2, then the
directory would set the owner to the remaining sharer, clear the sharers
list, send an eviction ack to the requester, and transition to {}``Exclusive.''

When a shared read arrives in the {}``Shared'' state, the directory
would add the requester into the sharers list, forward the request
to a sharer S, which can be any node among the available sharers,
and transition to {}``Shared Waiting for Reply.'' In this state,
the directory adds any new shared requesters into the sharers list
and naks any exclusive reads and eviction requests that are not from
sharer S. If the directory receives a shared reply from sharer S in
this state, it would send shared replies to all requesters. However,
if an eviction request from sharer S was received, then the directory
would send shared replies to all read requesters, send an eviction
ack to the eviction requester, remove the eviction requester from
the sharers list, and transition back to {}``Shared.''

While in {}``Shared'' state, it is also possible to receive an exclusive
read request. This request causes the requester to be set as the owner,
a transition to {}``Waiting to Send Invalidates,'' and the request
to be forwarded to a sharer S, which can be any node among the available
sharers. In the {}``Waiting to Send Invalidates'' state, if the
directory receives an eviction request not from sharer S, it would
send a nak to the requester. Likewise, if the directory receives a
read request in this state, it would send a nak to the requester.
If a read reply arrives from sharer S, the directory would send invalidates
to all sharers and transition to {}``Waiting for K Invalidates, J
Invalidates Received So Far,'' where K is the number of invalidates
sent, and J is the number of invalidate acks received so far. However,
if an eviction request from sharer S arrives, then the directory would
remove the requester from the sharers list, send invalidates to the
remaining sharers, send an eviction ack to the requester, then transition
to {}``Waiting for K Invalidates, J Invalidates Received So Far.''

The directory's primary purpose is to wait for invalidate acks in
{}``Waiting for K Invalidates, J Invalidates Received So Far.''
If a read request or an eviction request was received, the directory
has to send a nak to the requester. If an invalidate ack was received,
the directory checks if J (the number of invalidate acks received
so far) is equal to K (the number of invalidates sent). If they are
not the same, then the directory continues to wait in this state.
However, if J = K, then the directory would send an exclusive reply
to the requester and transition to {}``Exclusive.''


\subsubsection{Regular Protocol Cache FSM}

\figurename~\ref{fig:Regular-protocol-cache}illustrates the cache
FSM for the regular directory-based cache-coherence protocol. All
caches start initially in the {}``Invalid'' state. From this state,
the cache can receive a request from the processor for either a shared
read or an exclusive read. If the cache receives a shared read, then
it would transition to {}``Waiting for Shared Read Reply'' after
forwarding the request to the directory. However, if the cache receives
an exclusive read, it would transition to {}``Waiting for Exclusive
Reply'' after forwarding the request to the directory.

%
\begin{figure}
\includegraphics[width=1\textwidth]{diagrams/regular-cache-fsm-1}

\caption{Regular protocol cache FSM\label{fig:Regular-protocol-cache}}
%
\end{figure}


In {}``Waiting for Shared Read Reply,'' the cache waits for either
type of read replies and transitions depending on which type of read
reply it receives. If it receives a nak, it has to resend the shared
read request. If the cache receives a shared reply, it would fill
the cache with the block and transition to {}``Shared.'' However,
if the cache receives an exclusive reply, it would transition to {}``Clean
Exclusive,'' instead.

Once in {}``Clean Exclusive,'' the cache can receive an invalidate
message from the directory, when it has to reply with an invalidate
ack and transition to {}``Invalid.'' In this state, the cache could
also decide to evict the block by sending an eviction request to the
directory and transitioning to {}``Clean Exclusive Waiting for Eviction
Ack,'' where it waits for an eviction ack before transitioning to
{}``Invalid.'' Otherwise, the cache can also transition into the
{}``Dirty Exclusive'' state if it receives an exclusive read request
from the processor.

After arriving in the {}``Dirty Exclusive'' state, several things
can happen. The cache can receive an invalidate from the directory,
when it would have to transition to {}``Invalid'' and send an invalidate
response with the block attached to the directory. If the cache has
to evict the block, it would send a writeback request to the directory
and transition into {}``Dirty Exclusive Waiting for Writeback Ack,''
where it waits for a writeback ack from the directory before transitioning
to {}``Invalid.'' In {}``Dirty Exclusive,'' the cache could also
receive a shared read request from the directory, indicating that
the cache should transition into the {}``Shared'' state.

Once the cache transitions into the {}``Shared'' state, several
transitions could happen. The directory could \ receive an invalidate
message from the directory, indicating that the cache should transition
to {}``Invalid'' and send an invalidate ack to the directory. If
this cache received an invalidate, then this cache could not have
been the sharer S that the directory chose to read from, since the
directory is required to obtain the first read reply from sharer S
before sending out its invalidates. In addition, if the cache has
to evict a block in the {}``Shared'' state for any reason, it would
send an eviction request to the directory and transition to {}``Shared
Waiting for Eviction Ack.''

After transitioning to {}``Shared Waiting for Eviction Ack,'' the
cache has to wait for one of two messages. If the cache receives an
invalidate message from the directory, then it means that the eviction
request has not arrived at the directory before the invalidate message
was sent, so it has to transition to {}``Waiting for Nak,'' since
the directory would send a nak in response to the cache's eviction
request. Once the cache receives a nak in {}``Waiting for Nak,''
it would send an invalidate ack and transition to {}``Invalid.''
It is also possible for the cache to receive an eviction ack while
in {}``Shared Waiting for Eviction Ack,'' where the cache can transition
to {}``Invalid'' without doing anything else.

In {}``Shared,'' it is also possible for the cache to receive an
exclusive read request from the processor, where it would forward
the request to the directory and transition to {}``Waiting for Exclusive
Read Reply.'' In this state, the cache can receive a nak from the
directory, telling the cache that the directory is busy and that it
should resend its request. Only the exclusive reply message from the
directory will transition the cache out of this state and into the
{}``Dirty Exclusive'' state.


\subsection{Origin Protocol}

The Origin directory-based cache-coherence protocol uses a three-way
interaction that sends less messages than the regular directory protocol
does. Because of this change, the finite-state machines for this protocol
is substantially different than the regular protocol, both in the
directory and in the cache.


\subsubsection{Origin Protocol Directory FSM}

\figurename~\ref{fig:Origin-protocol-directory} illustrates the
directory FSM for the Origin directory-based cache-coherence protocol
and shows all the states in a directory. Any block will start in the
{}``Unowned'' state. From here, if the directory receives either
type of read request, the directory would transition to {}``Exclusive
(Memory-Access),'' make requester owner, and perform memory fetch.
If the directory receives another read request in {}``Exclusive (Memory-Access),''
the directory would send a nak to the requester. When the request
from memory returns in the {}``Exclusive (Memory-Access)'' state,
the directory would send an exclusive reply to the requester and transition
to the {}``Exclusive'' state. It should be impossible for a writeback
request to arrive during {}``Unowned'' or {}``Exclusive (Memory-Access)''
states because no node should have the block in these two states.

%
\begin{figure}
\includegraphics[width=1\textwidth]{diagrams/origin-fsm-directory-03-combined}

\caption{Origin protocol directory FSM\label{fig:Origin-protocol-directory}}
%
\end{figure}


From the {}``Shared'' state, there are several states the directory
can transition into. The first is {}``Shared (Memory-Access)'',
which occurs if a shared read request comes in while in the {}``Shared''
state. During the transition, the directory would perform a memory
fetch and add the requester to sharers. In {}``Shared (Memory-Access)'',
the directory would queue up any more shared read requests that might
come in while sending a nak to any exclusive reads. When the memory
returns, the directory will send shared replies to all requesters.
There could also be a transition to the {}``Shared Exclusive (Memory-Access)''
state or the {}``Exclusive'' state from the {}``Shared'' state
when it receives an exclusive read request, depending on whether or
not the requester can be found in the owner or sharers of the directory
block. If the requester is the owner or in sharers, then the directory
has received an upgrade request, which means that the requester already
has the data. Otherwise, the directory has to fetch the block from
memory before it can send out invalidates to sharers and previous
owner, send out exclusive reply with invalidates pending to the requester,
set the requester as the owner, and clear sharers. It is impossible
to receive a writeback request in the {}``Shared'' state or any
of the states that transitions from the {}``Shared'' state because
no node has exclusive access to the block. If more read requests arrive
while in the {}``Shared Exclusive (Memory-Access)'' state, then
the directory has to send a nak to that request.

In the {}``Exclusive'' state, the home memory could transition to
the {}``Exclusive (Memory-Access)'' state if the requester is the
owner, while also performing a fetch from memory. It could also transition
to {}``Unowned'' if it receives a writeback request. However, from
{}``Exclusive,'' it could also transition to the two busy memory-access
states.

In all of the busy states, naks are sent in response to any requests,
since the directory cannot process any additional requests when it
is busy. The directory will transition to the busy memory-access states
when a read request arrives during the {}``Exclusive'' state and
the owner is not the requester. It is necessary, both when transitioning
to {}``Busy-Shared (Memory-Access)'' and when transitioning to {}``Busy-Exclusive
(Memory-Access),'' to send a memory fetch and set the owner as the
requester; however, it is only necessary to set the sharers as the
previous owner when transitioning to {}``Busy-Shared (Memory-Access)''
since {}``Busy-Exclusive (Memory-Access) transitions eventually to
the {}``Exclusive'' state. In the {}``Busy-Shared (Memory-Access)''
state, we usually would wait for the memory to return before performing
any other operations. If the directory receives a memory return first,
then it would send an intervention request to the previous owner,
send a speculative reply to the requester, and transition to the {}``Busy-Shared''
state. Should a writeback request occur before the memory return,
however, the directory would transition to the {}``Busy-Shared (Memory-Access
Writeback-Request)'' state, where we would wait for the memory return
and not send any messages that we would send when we transition to
the {}``Busy-Shared'' state. In the {}``Busy-Shared (Memory-Access
Writeback-Request)'' state, we would wait for the memory return message,
then send a regular shared reply to the owner and a writeback exclusive
ack to the requester. Another writeback request could not arrive while
in the {}``Busy-Shared (Memory-Access Writeback-Request)'' state
because the directory just received a writeback request from the only
node that had the block. The {}``Busy-Exclusive (Memory-Access)''
state transitions are similar, except we send exclusive messages instead
of shared messages and we transition to exclusive states.

In the {}``Busy-Shared'' state, the directory is waiting for a message
from the previous owner of the block before transitioning to the {}``Shared''
state. This could come in the form of a writeback request (if the
owner evicted the block before the intervention arrives), a shared
writeback, or a shared transfer (no data). If a writeback request
or a shared writeback arrives, the directory also has to write the
block to memory. Additionally, a writeback request means that the
directory would forward that data to the new owner and return a writeback
busy ack to the requester. It is possible for the intervention to
be unsuccessful, however, which is indicated by a nak from the previous
owner. When this happens, the directory would return to {}``Exclusive''
while setting the owner to the previous owner and clear the sharers.
The {}``Busy-Exclusive'' state is similar, except exclusive messages
are sent instead of shared messages, the directory transitions back
to {}``Exclusive'' upon successful completion, and upon receiving
a nak in {}``Busy-Exclusive,'' it is not necessary to clear the
sharers.


\subsubsection{Origin Protocol Cache FSM}

\figurename\ \ref{fig:Origin-protocol-cache} illustrates the cache
FSM for the Origin directory-based cache-coherence protocol. This
diagram shows the state transitions for the cache. Any cache line
for an address starts in the {}``Invalid'' state. If the cache receives
a shared read from the processor, it transitions to {}``Waiting for
Shared Read Response'' and forwards the request to the home memory;
if it receives an exclusive read instead, it transitions to {}``Waiting
for Exclusive Read Response.'' In addition, if the cache receives
an intervention, it sends an ack to the requester and a transfer to
the directory. The type of the ack and the transfer will be dependent
upon whether or not a shared intervention is received or an exclusive
intervention is received. If the cache receives an invalidate in the
{}``Invalid'' state, then it means that the cache has evicted the
block because of a capacity miss, which means that there is no need
to invalidate the block again because the cache does not have the
block. However, it is still necessary to send an invalidate ack to
the requester in accordance with the protocol.

%
\begin{figure}
\includegraphics[width=1\textwidth]{diagrams/origin-fsm-cache-02}

\caption{Origin protocol cache FSM\label{fig:Origin-protocol-cache}}


%
\end{figure}


Once the cache transitions to {}``Waiting for Shared Read Response,''
it is possible for the cache to receive a nak from the directory,
indicating that the directory is currently busy and cannot handle
the request, and the cache will have to resend its request. If it
receives a shared reply, the cache would transition to the {}``Shared''
state, while receiving an exclusive reply would transition the cache
into {}``Clean Exclusive.'' If the cache receives an intervention
message while in the {}``Clean Exclusive'' state, it has to send
an ack to the requester, and send a transfer to the directory, before
transitioning to {}``Shared'' or {}``Invalid,'' depending on whether
or not it was a shared intervention or an exclusive intervention.
It is also possible, while in {}``Clean Exclusive,'' to receive
an exclusive read request, in which case the cache would transition
to {}``Dirty Exclusive.'' Finally, if the cache needs to evict a
block while in {}``Clean Exclusive,'' it would transition to {}``Invalid''
without sending any messages.

However, if a cache receives a speculative reply or a shared response/ack
in {}``Waiting for Shared Read Response,'' then it would transition
to {}``Waiting for Shared Response/Ack'' or {}``Shared Waiting
for Speculative Reply,'' respectively. Receiving either of these
messages mean that the directory block state was Exclusive and that
the requester needs to wait for both messages before it can fill its
cache and transition to the {}``Shared'' state. However, it is possible
to get a nak from the previous owner while in the {}``Waiting for
Shared Response/Ack'' state, in which case, the cache has to start
all over and resend the request while transitioning back to {}``Waiting
for Shared Read Response.''

In the {}``Shared'' state, if the cache receives an invalidate from
the directory, the cache would have to invalidate the block, send
an invalidate ack to the requester, and transition to the {}``Invalid''
state. If the cache needs to evict the block, it would also have to
transition to {}``Invalid,'' but without sending any messages. It
is also possible to receive an exclusive read request from the processor
in this state, when the cache would have to forward the request to
the directory and transition to {}``Waiting for Exclusive Read Response.''
Since the cache might have received the exclusive read request before
receiving an invalidate from the directory, it is necessary to send
an invalidate ack to the requester when receiving an invalidate in
the {}``Waiting for Exclusive Read Response'' state. If the cache
receives a nak in {}``Waiting for Exclusive Read Response,'' the
directory could not fulfill the request and the cache must resend
its request. Typically, the cache would receive an exclusive reply
in this state and transition to {}``Dirty Exclusive.''

However, if there were any sharers when the cache requested exclusive
read, the cache would receive an exclusive reply with invalidates
pending or an invalidate ack and transition to {}``Waiting for K
Invalidates, J Invalidates Received So Far.'' The reason that an
exclusive reply with invalidates pending or an invalidate ack could
be received from the {}``Waiting for Exclusive Read Response'' state
is illustrated in \figurename\ref{fig:Scenarios-When-Transitioning}.
The exclusive reply with invalidates pending and invalidations are
sent at the same time to the requester and the sharers, respectively,
in both scenarios. In the expected case, scenario 1, the exclusive
reply with invalidates pending would arrive at the requester before
any invalidate acks. However, if any sharers are on the same node
as the directory, it can receive its invalidate before the exclusive
reply with invalidates pending arrives at the requester. The sharer
would then send an invalidate ack to the requester. Since the invalidate
ack has to traverse the network, there is no guarantee that the exclusive
reply with invalidates pending would arrive first. Therefore, the
requester has to prepare for both scenarios by transitioning into
the {}``Waiting for K Invalidates, J Invalidates Received So Far''
state on the arrival of either types of messages. In this state, the
cache waits for the arrival of invalidate acks and/or exclusive reply
with invalidates pending and transitions to {}``Dirty Exclusive''
only when all invalidate acks and the exclusive reply with invalidates
pending have been received. It is also possible to receive an intervention
while in {}``Waiting for K Invalidates, J Invalidates Received So
Far,'' when the cache would have to respond with a nak to the directory
and a nak to the requester.

%
\begin{figure}
\includegraphics[width=1\textwidth]{diagrams/waiting-for-k-invalidates-j-invalidates-received}

\caption{Scenarios When Transitioning into {}``Waiting for K Invalidates,
J Invalidates Received So Far''\label{fig:Scenarios-When-Transitioning}}


%
\end{figure}


In the {}``Waiting for Exclusive Read Response'' state, it is also
possible that the cache has to wait for two messages before it can
transition to {}``Dirty Exclusive.'' When the cache receives an
exclusive response/ack or a speculative reply, it would transition
to {}``Exclusive Waiting for Speculative Reply'' or {}``Waiting
for Exclusive Response/Ack,'' respectively, while it waits for the
other message of this two-message exchange to arrive. When the other
message arrives, the cache would fill the cache with the exclusive
response (if exclusive response) or speculative reply (if exclusive
ack) and transition to {}``Dirty Exclusive.'' However, it is possible
that the cache receives a nak from the previous owner in the {}``Waiting
for Exclusive Response/Ack'' state, when the cache would have to
start over by resending its request to the directory and transitioning
to {}``Waiting for Exclusive Read Response.''

Once in the {}``Dirty Exclusive'' state, the cache can receive three
types of messages. If the cache receives a shared intervention, it
would send a shared response to the requester and send a shared writeback
to the directory while transitioning to {}``Shared.'' Similarly,
if the message was an exclusive intervention, it would send an exclusive
response to the requester and a dirty transfer to the directory while
transitioning to {}``Invalid''. Furthermore, the cache can also
receive a writeback request from the processor in this state, where
it has to forward the request to the directory and transition to {}``Waiting
for Writeback Response.''

If the cache receives a simple writeback exclusive ack in {}``Waiting
for Writeback Response,'' the cache can transition to {}``Invalid''
without any other operations. If the cache receives an intervention,
it would transition to {}``Waiting for Writeback Busy Ack,'' where
it would wait for a writeback busy ack before transitioning to {}``Invalid''.
Similarly, if the cache receives a writeback busy ack first, it would
transition to {}``Waiting for Intervention'' and wait for the intervention
before it can transition to {}``Invalid''. It is also possible that
the directory is in one of the busy states while the cache is {}``Waiting
for Writeback Response,'' and the cache has to resend its writeback
request if that happens.


\section[Methodology and Implementation]{Methodology and Implementation}

The implementation of the code was done SESC Simulator, which uses
C++ as its programming language. It supports various features of a
computer system that allows us to simulate the differences between
different directory-based protocols.


\subsection{Directory}

In the simulator, we make sure that whenever a message is being sent
to a remote node that has the same node ID as the current node, we
send the message to itself. For example, if processor 1 requests a
block A, where processor 1 is block A's home node, sending a network
message from a node to itself it pointless and just results in needless
network traffic. We perform a check to see if the destination for
a message is the same as the source of the message, previous to sending
the network message. If this is the case, it just forwards it straight
on to the function that would handle it on the local directory node
rather than emitting something on the network.


\subsection[Cache]{Cache}

Since directories and buses are basically ways for multiple caches
to communicate with each other, it is important to understand what
caches are and what kind of requests they can send to each directory.
{}``In a coherent multiprocessor, the caches provide both migration
and replication of shared data items'' \cite{HEN00}. It is important
for the architect of the processor to design these features into the
processor as to allow the programmer to take advantage of the speedup
available in having multiple data. Having multiple data allows for
multiple reads at the same time. The two protocols introduced here
serves as an example of some of the protocols that can be used to
achieve cache coherency. The MSI protocol is the most basic one and
only uses three states. The MESI protocol adds an {}``Exclusive''
state to reduce the traffic caused by writes of blocks that only exist
in one cache. The MOSI protocol adds an {}``Owned'' state to reduce
the traffic caused by write-backs of blocks that are read by other
caches. The MOESI protocol implements both the Exclusive and Owned
state to take on both characteristics.


\subsubsection[Changing Cache Size]{Changing Cache Size}

%
\begin{table}[htbp]
\begin{tabular}{|>{\raggedright}m{0.1\textwidth}|>{\raggedright}m{0.2\textwidth}|>{\raggedright}m{0.2\textwidth}|>{\raggedright}m{0.2\textwidth}|>{\raggedright}m{0.2\textwidth}|}
\hline 
 & Main Memory  & Cache Line  & Other Processor  & Snoop Request \tabularnewline
\hline 
Modified  & Stale  & Most recent, correct copy  & Hold no copy  & Modified cache responds \tabularnewline
\hline 
Exclusive  & Most recent, correct copy  & Most recent, correct copy  & Hold no copy  & Exclusive cache or main memory responds \tabularnewline
\hline 
Shared  & Can hold most recent, correct copy  & Can hold most recent, correct copy  & Shared or owned  & May not respond \tabularnewline
\hline 
Invalid  & May hold valid or invalid  & Invalid copy  & May hold valid or invalid copy  & May not respond \tabularnewline
\hline
\end{tabular}\caption{Acceptable states for MESI Cache}
%
\end{table}


The cache size can be determined by associativity {*} number of sets
{*} width. Therefore, in the configuration file, it is only necessary
to give these three parameters and not the total size of the cache.
For example, in my system, this could be 4 {*} 4 {*} 64, in the case
of the L1 cache. This gives it 16 blocks of 64-bit data to have a
1kB L1 cache. In the case of the L2 cache, I used an associativity
of 4 with 8 sets to get 32 blocks. this equates to a 2kB cache.

Increasing cache size increases coherence misses because more invalidates
occur because fewer blocks are bumped due to capacity misses. Of course,
capacity misses decrease because the cache has more spaces to put
blocks \cite{HEN00}. Increasing block size means capacity miss decreases
and compulsory miss decreases for certain applications. When this
happens, it most likely means that there is a lot of spatial locality
in the code, such as when running kernel code. Because increasing
block size grabs more of the code in the same area together, which
directly reduces compulsory misses. The capacity miss is reduced because
we're storing more of the necessary code in the cache \cite{HEN00}.


\subsection[Network]{Network}

The underlying network in use for this simulator is a simple black
box model. It does not model a real network with routing. Instead
it models messages going in and out of the network using a random
delay with a lower-bound of 4 and an upper-bound of 20. When more
messages arrive, the delay coming from the random delay generator
will be shifted higher as to model the higher traffic conditions.

We could have used a more complicated network, one that simulates
router-router connection and uses routing protocols, but such a network
was not available at the time in the simulator, and the black box
network serves its purpose for delivering messages to and from all
the directories, as well as the main memory. Although the SESC simulator
has the capability to model a more complicated network, a simple network,
such as the one used here, can illustrate our point.


\subsection[SESC Simulator]{SESC Simulator}

The connections in the SESC Simulator are shown in \figurename {[}still
have to add this{]}. We are using an architecture with a 1K L1 cache
and a 2K L2 cache.


\section[Results]{Results}

In this section, I will discuss the results of running the simulation
on the Sesc simulator. The results are shown in Table 2 for the results
using the benchmark fft in the SPLASH-2 benchmark suite with 32 processors.
The fft benchmark is a complex 1-D version of the radix-${\surd}$n
six-step FFT algorithm, which is optimized to minimize inter-processor
communication \cite{WOO00}.


\subsection[Verifying Correctness]{Verifying Correctness}

To verify whether or not the simulator is correct, we run the benchmark
on a normal machine to find out what the output is, then run the program
on SESC. The output produced from SESC should be identical to that
produced by running the benchmark on a real processor. If not, then
it means that the program that simulates the directory protocol is
not running correctly. In addition, the SPLASH2 benchmarks' kernel
programs provide self-test that we can invoke to ensure that our protocols
were implemented directly. It achieves this self-test using inherent
tests to the data structure.


\subsection[SPLASH2 benchmarks]{SPLASH2 benchmarks}

Using these results, we can see that the MSI protocol takes longer
to complete. Although there are more read misses for the MOESI cache,
it is offset by the amount of read hits for L2 cache. \ It could
also be possible that the additional read misses do not incur as much
penalty in a MOESI cache because in a MOESI cache, there is a greater
opportunity for the cache to fetch the data from another node in that
directory instead of from the memory.

We also see that for both MOESI and MSI caches, as the CPU count increases,
the run-time decreases, this is to be expected as more work can be
done in the same amount of time when one has more processors that
can be used at the same time.


\section[Problems]{Problems}

In implementing the directory-based cache-coherence protocol, there
were some problems. One was simply that debugging such a large system
is inherently hard, especially since the bugs often surface after
tens of thousands of messages are sent. It is useful to print out
each message that pass around the system. To debug these problems,
it is useful to know which messages cause which transitions. For example,
if we see that a message is being sent to the directory node when
it should be sent to the requesting node, we know that there is a
problem. The debugging system works by various lines of assertions
that should always hold true if the system is working. They can easily
be turned off to increase speed of the simulator by not defining DEBUG
when compiling.


\subsection{Further Work}

The memory system in the current system is implemented in a way such
that it is in one location. In the future, we could modify the the
simulator such that the memory system is distributed just like the
directory system. With a distributed memory system, there would be
much more issue with coherence, scaling, performance, and correctness.

We can compare the protocol we have to a snarfing protocol. In this
protocol, the cache controller watches both address and data in an
attempt to update its own copy of a memory location when a second
master modifies a location in main memory. When a write operation
is observed to a location that a cache has a copy of, the cache controller
updates its own copy of the snarfed memory location with the new data.\cite{FAR00}\cite{LAU00}


\section{Conclusion}

We have arrived at the conclusion that the Origin-based directory
protocol is superior, since a few of its transactions can be done
using a three-way interaction instead of two bilateral interactions.
\bibliographystyle{plain}
\bibliography{report2}
 
\end{document}
